% background.tex
\documentclass[main.tex]{subfiles}
\begin{document}
\chapter{Background}
\section{Additive Manufacturing}\label{sec:AM} %Section labeling for cross-referencing
\emph{Additive Manufacturing} (AM) technologies had their beginnings in the decade of the 1980s. During this time, various independently developed patents were filed across the globe, describing a process that would construct an object by selectively adding layers of material -as opposed to removing excess matter or deforming mass to obtain a desired shape. This represents the core definition of AM: any technology where the final geometry of the manufactured object is obtained through controlled addition of material qualifies as an Additive Manufacturing technique \cite{Gibson2015}.

Advancements in the fields of computing, \emph{Computer Aided Design} (CAD), and controllers, among other technological developments, were necessary to translate the patents into working prototypes, with some eventually becoming the foundations of commercially successful companies -such as 3D Systems in 1986 and Stratasys in 1989 \cite{Gibson2015,3DSystems,Stratasys2017}. However, the basic process of AM has remained largely unchanged from its first iteration in the late 80s: First, a computer model of the object is made using CAD software and exported under the .\emph{stl} file format. Afterwards, the part geometry is stratified, or \textquotedblleft sliced\textquotedblright, and translated into machine instructions using a specialized software called \emph{slicing engine}. An AM machine then follows said instructions, commonly referred to as the \emph{toolpath}, to build the object in layers. Finally, the part is available to the user. Depending on either the requirements of the part, or the specifics of the AM technique used, some post-processing may be required \cite{Gibson2015}. A visual representation of the process is shown in Figure~\ref{fig:AM_flow}.

\begin{figure}[h]
	\center
	\includegraphics[width=\linewidth]{AM_flowchart_1}
	\caption{Process flow of AM} \label{fig:AM_flow}
\end{figure}
\pagebreak %Used to move the entire paragraph to a new page.
 
While all AM technologies operate on the same basic process flow described above, the specifics of each AM technique vary substantially, ranging from processes that use paper and binder, all the way through metal-based, laser tracing technologies. Since this is a rapidly evolving field, no general consensus exists for classifying the multiple AM processes available. However, the classification system proposed under the ASTM/ISO 52900 standard \cite{ASTM52900}, has been somewhat accepted by the field and divides AM technologies as follows:
\begin{enumerate}
	\item \textbf{Binder Jetting}: AM techniques where a binding agent is used to selectively promote cohesion in powder materials -generally gypsum, sand or metallic powders~\cite{ASTM52900,3DHubs2018}.
	\item \textbf{Directed Energy Deposition}: AM processes where a focused thermal energy source (i.e. laser, electron beam, plasma arc) is used to fuse materials as they are being deposited in the build volume. Materials are almost exclusively metals~\cite{ASTM52900,3DHubs2018}.
	\item \textbf{Material Extrusion}: In this type of AM technology, material is dispensed through a nozzle or orifice. Fused Filament Fabrication belongs to this classification. Materials are almost exclusively thermoplastics \cite{ASTM52900,3DHubs2018}.
	\item \textbf{Material Jetting}: AM techniques where build material is deposited selectively in droplets. Materials are usually wax or thermoplastics, but there are examples of metal-based, material jetting techniques \cite{ASTM52900,3DHubs2018}.
	\item \textbf{Powder Bed Fusion}: AM processes where portions of a powder bed are selectively fused through application of thermal energy. \emph{Selective Laser Sintering} (SLS) belongs to this category. Materials are usually thermoplastic polymers or metals \cite{ASTM52900,3DHubs2018}. 
	\item \textbf{Sheet Lamination}: In this type of AM technology, the final part is formed by bonding sheets of material -usually paper or composites \cite{ASTM52900,3DHubs2018}. 
	\item \textbf{Vat Photopolymerization}: In this AM process, a liquid photopolymer is selectively cured by a light source. \emph{Stereolithography} (SLA), arguably the first AM technology, belongs to this category. Due to the nature of this technique, the only materials used are photopolymers \cite{ASTM52900,3DHubs2018}.
\end{enumerate} 

\subsection{Advantages, Disadvantages and Success Stories}\label{subsec:AMAdDis} 
Since AM processes allow a relatively direct conversion of a CAD model into a constructed object, they were originally exclusively used for prototype development. For this reason, they were initially classified as \textquotedblleft \emph{Rapid Prototyping}\textquotedblright~(RP) technologies. This terminology is still used today, however, it is being superseded by \emph{Additive Manufacturing} since its potential to become a proper fabrication technique exists \cite{Gibson2015}. However, while being capable of quickly jumping from part design to manufacturing is a great advantage, AM has its own set of drawbacks. Table \ref{tab:AM_AdDis} summarizes the most noteworthy set of advantages and disadvantages typical of most AM technologies.

\begin{table}[h]
	\centering
	\caption{Advantages and Disadvantages of Additive Manufacturing}
	\label{tab:AM_AdDis}
	\begin{tabu} to 0.95\textwidth {  X[c]  X[c] }
		\hline
		\textbf{Advantages} & \textbf{Disadvantages} \\ 
		\hline
		Faster product development cycles \cite{Gibson2015} & Part quality highly dependent on process parameters \cite{Gibson2015}\\
		%---------
		No additional tools needed for part fabrication\cite{Gibson2015}&  Stratified build generally results in anisotropic parts \cite{Gibson2015, Capote2017}\\
		%---------
		Cost effective for small batches of parts \cite{Baumers2016,Conner2014,Berman2012}&  Costly for production of more than hundreds of parts \cite{Baumers2016,Conner2014,Berman2012}\\
		\hline
	\end{tabu}
\end{table}   

Out of all the described advantages and disadvantages, the high anisotropy of AM parts is responsible for the slow embrace of AM in highly demanding engineering fields -such as the aerospace and automotive industries. The highly anisotropic mechanical behavior makes it extremely difficult to predict part failure, therefore, it can't be implemented in engineering applications where catastrophic failure is to be avoided at all costs. Even so, success stories of implementation of AM in industrial environments are abundant. Below is a number of relatively 
recent examples:

\begin{itemize}
	\item \textbf{Volkswagen Autoeuropa}: This automotive assembly plant implemented the use of FFF machines to manufacture tools, jigs and fixtures used in their assembly line. They now produce 93\% of the tools that were historically externally sourced, and have reportedly cut their tool development time and costs by 95\% and 91\% respectively \cite{deVries2017}.
	\item \textbf{General Electric}: GE is currently producing in Alabama a complex fuel nozzle injector for the LEAP jet engine, using powder based, metal AM. The complex geometry of this component could not be manufactured by any other manufacturing technique. The production plant is expected to have 50 AM machines producing 35,000 fuel nozzle injectors annually by 2020 \cite{GEAdditive2016}. 
	\item \textbf{Adidas and New Balance}: Both shoe companies have developed separate approaches to constructing highly optimized, 3D printed midsoles for high performance running sneakers. New Balance makes use of SLS technology to build the intricate geometry of their \textquotedblleft \emph{Zante Generate}\textquotedblright~sneaker, using powdered TPU elastomer as the parent material. The designed honeycomb structure of the midsole, combined with the flexible material used, are supposed to improve the comfort and support brought by the shoe \cite{NewBalance2016}. Adidas on the other hand chose to develop the \textquotedblleft \emph{AlphaEDGE 4D LTD}\textquotedblright~running shoe using the CLIP technology by Carbon3D. While the cell geometry in  the midsole is also supposed to bring performance and comfort improvements, the final ambition of Adidas is to perfect the technology to a point where a customer can simply go to a shoe store, have their feet scanned, and receive a fully customized shoe with a 3D printed midsole that fits their particular needs \cite{Matisons2015,Saunders2018}. In both cases, the geometry of the midsole can only be produce by AM. The intricate structures in the midsoles can be seen in Figure \ref{fig:AMshoes}.
		
\end{itemize}
\begin{figure}[h]
	\center
	\subfloat[New Balance Zante Generate~\cite{NewBalance2016}\label{fig:NBZG}]{%
		\includegraphics[height=6cm, keepaspectratio]{NB_generate}
	}
	\hfill
	\subfloat[Adidas AlphaEdge 4D LTD~\cite{Saunders2018}\label{fig:adidas}]{%
		\includegraphics[height=6cm, keepaspectratio]{alphaedge_4D}
	}
	\caption{Shoes with AM midsoles}
	\label{fig:AMshoes}
\end{figure}

Note that in the cases presented, the main reason behind the usage of AM was either reduction expenses associated with producing small batches of parts, or the capability of reproducing a unique and complex geometry. This is a trend that is observed in most of the literature describing implementation of AM into industrial scenarios.

While the advantages and disadvantages described here cover the field of AM as a whole, each technique comes with its own set of pros and cons that may make it the preferred method to reproduce a particular product or geometry. This work, however, focuses solely on FFF. The specifics of this process are described in detail in Section~\ref{sec:FFF}.
\section{Fused Filament Fabrication}\label{sec:FFF} 
\emph{Fused Filament Fabrication}~(FFF) is an AM technology where the final geometry of the part is obtained through controlled extrusion of a liquid, self-hardening material -usually a thermoplastic polymer in molten state \cite{Gibson2015}. Originally developed by Stratasys in the 1980s under the \textendash~still trademarked \textendash~\emph{Fused Deposition Modeling}~(FDM\texttrademark) moniker, it has recently become one of the most widely used AM techniques due to the advent of low-cost, desktop FFF machines in the early 2010s caused by the expiration of key patents from Stratasys \cite{Gibson2015,Capote2017}. 

\pagebreak
\subsection{The FFF process}\label{ssec:FFFmach}
Like all AM technologies, the FFF process starts in a computer with a CAD model converted to the \emph{.stl} file format. The geometry is then converted to machine instructions through a \emph{slicing engine}, where the user inputs process parameters. Finally the \emph{toolpath} is executed by the FFF printer, building the object in a layer-by-layer basis -sometimes referred to as \emph{2.5D} printing. Figure \ref{fig:FFFflow} shows an abridged version of the process. The \emph{z} axis indicates the intended build direction. 
\begin{figure}[h]
	\center
	\includegraphics[width=0.95\textwidth]{FFF_flow}
	\caption{From left to right: stl, toolpath and final part} \label{fig:FFFflow}
\end{figure}
\begin{figure}[h]
	\center
	\includegraphics[height=4.5cm]{nozzle}
	\caption{FFF printhead} \label{fig:FFFnoz}
\end{figure}
%Nomenclature introduced in this chapter:
\nomenclature[A]{SLA}{Stereolithography}% 
\nomenclature[A]{SLS}{Selective Laser Sintering}%
\end{document}