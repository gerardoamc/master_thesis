% motivation.tex
\documentclass[main.tex]{subfiles}
\begin{document}
\chapter{Motivation and Objective}

%Chapter body
Additive Manufacturing (AM) is an umbrella term that encompasses all fabrication techniques where the final geometry of the part is obtained through superposition of material in a layer-by-layer basis \cite{Gibson2015}. Developed in the 1980s, this manufacturing technique permits immensely shorter part development cycles since the transition from a 3D computer aided design (CAD) to part fabrication only requires one intermediate step: the use of a slicing engine that converts the geometry of the object into machine instructions \cite{Gibson2015}. For this reason, AM technologies were initially employed exclusively for prototype development and were referred to as Rapid Prototyping techniques (RP). However, recent innovations in the field have caused AM to be perceived as a legitimate manufacturing technology since it is also capable of reproducing complex geometries unattainable through other means \cite{Gibson2015}.

While offering great advantages over traditional part fabrication methods, AM comes with its own set of limitations and disadvantages: First and foremost, the use of a stratified build approach tends to produce extremely anisotropic parts. Secondly, the geometric accuracy of the object produced is highly dependent of process parameters, particularly of the thickness of the layers. Finally, as of the time of this writing, AM lacks the standardization and scrutiny that are associated to most traditional manufacturing techniques \cite{Gibson2015}.  

Fused Filament Fabrication (FFF), also known under the trademark Fused Deposition Modeling (FDM\texttrademark), represents perhaps the most prevalent AM technique in the market due to the advent of low-cost, desktop 3D printers in the early 2010s \cite{Capote2017}. Due to the broad availability of machines and relatively low costs of material, there's a surging interest in optimizing FFF to produce small batches of end user grade parts. Success stories include vacuum form molds, fixtures, jigs and tools used to aid assembly lines in the automotive industry \cite{Hartman2014, VanHulle2017,deVries2017}. However, this technology still faces the challenges and limitations that currently affect the field of AM as a whole. Namely, anisotropy introduced through the layer-by-layer build approach makes it difficult to assess the expected mechanical behavior of FFF produced parts when subjected to important stresses \cite{Capote2017}.   

%Nomenclature introduced in this chapter:
\nomenclature[A]{AM}{Additive Manufacturing}% 
\nomenclature[A]{RP}{Rapid Prototyping}%
\nomenclature[A]{CAD}{Computer Aided Design}%
\nomenclature[A]{FDM}{Fused Deposition Modeling\texttrademark}
\nomenclature[A]{FFF}{Fused Filament Fabrication}%

\end{document}