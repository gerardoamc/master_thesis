% abstract

\documentclass[main.tex]{subfiles}
\begin{document}
\setcounter{page}{1}
\chapter*{Abstract}
Fused Filament Fabrication (FFF) is arguably the most widely available Additive Manufacturing technology at the moment. Offering the possibility of producing complex geometries in a compressed product development cycle and in a plethora of materials, it comes as no surprise that FFF is attractive to multiple industries, including the automotive and aerospace segments. However, the high anisotropy of parts developed through this technique implies that part failure prediction is extremely difficult \textemdash a requirement that must be satisfied to guarantee the safety of the final user. For this reason, this work applies a novel criterion to define a failure surface that could prove an invaluable tool in formalizing the embrace of FFF in industry, since part failure prediction can finally be performed. Multiple mechanical tests are executed on coupons developed in a traditional FFF printer, as well as a customized, 6-axis robotic printer necessary to produce specimens in out of ordinary orientations. The results of these tests are used to populate the parameters of the mathematical function that describes the failure envelope. Results indicate strong interaction between axial loads and. %ELABORATE
 
\vspace{10mm} %5mm vertical space
\textbf{Keywords:} FFF, FDM, Failure Criteria, Off-axis printing.

\vfill
\begin{center}
Copyright~\textcopyright: Gerardo A. Mazzei Capote (2018)

\emph{All rights reserved}	
\end{center}
\end{document}