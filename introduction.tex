% introduction.tex

\documentclass[main.tex]{subfiles}
\begin{document}
\chapter{Introduction}

%Chapter body
\emph{Additive Manufacturing} (AM) is an umbrella term that encompasses all fabrication techniques where the final geometry of the part is obtained through superposition of material in a layer-by-layer basis \cite{Gibson2015}. Developed in the 1980s, this manufacturing technique permits immensely shorter part development cycles, since the transition from a 3D \emph{Computer Aided Design} (CAD) to part fabrication only requires one intermediate step: the use of a slicing engine that converts the geometry of the object into machine instructions \cite{Gibson2015}. For this reason, AM technologies were initially employed exclusively for prototype development and were referred to as \emph{Rapid Prototyping techniques} (RP). However, recent innovations in the field have caused AM to be perceived as a legitimate manufacturing technology since it is also capable of reproducing complex geometries unattainable through other means \cite{Gibson2015}.

While offering great advantages over traditional part fabrication methods, AM comes with its own set of limitations and disadvantages: First and foremost, the use of a stratified build approach tends to produce extremely anisotropic parts. Secondly, the geometric accuracy of the object produced is highly dependent of process parameters, particularly of the thickness of the layers. Finally, as of the time of this writing, AM lacks the standardization and scrutiny that are associated to most traditional manufacturing techniques \cite{Gibson2015}.  

\emph{Fused Filament Fabrication} (FFF), also known under the trademark \emph{Fused Deposition Modeling} (FDM\texttrademark), represents perhaps the most prevalent AM technique in the market due to the advent of low-cost, desktop 3D printers in the early 2010s \cite{Capote2017}. Due to the broad availability of machines and relatively low costs of material, there's a surging interest in optimizing FFF to produce small batches of end-user grade parts. Success stories are varied, but examples include vacuum form molds, fixtures, jigs, and tools used to aid assembly lines in the automotive industry \cite{Hartman2014, VanHulle2017,deVries2017}. However, this technology still faces the challenges and limitations that currently affect the field of AM as a whole. Namely, anisotropy introduced through the layer-by-layer build approach makes it difficult to assess the expected mechanical behavior of FFF parts when subjected to important mechanical stresses \cite{Capote2017}. For these reasons, multiple attempts have been made to characterize the anisotropy of FFF objects. Recent studies performed by Koch \emph{et al.} \cite{Koch2017} and Rankouhi \emph{et al.} \cite{Rankouhi2016} show that the ultimate tensile strength of FFF coupons is sensitive to process parameters such as the layer thickness and, in particular, the orientation in which the plastic strands are laid during the build process -henceforth referred to as the bead orientation. However, literature related to preventing failure through design is scarce, given the difficulty of using commercially available FFF machines to produce test coupons with unconventional bead orientations, as well as the limitations inherent to commonly used failure criteria that make it difficult to develop an accurate failure surface.

This research applies a novel criterion, tailored for anisotropic materials, to develop a failure surface for FFF parts through mechanical testing of coupons under various types of loading conditions. Certain test specimens were produced using a unique, in house developed off-axis 3D printer that allowed production of coupons in unconventional configurations. Such surface can be an invaluable tool in part design, since catastrophic failure can be prevented in the early stages of part development. This could potentially allow a broader embrace of FFF as a legitimate manufacturing technique in highly demanding engineering fields, such as the aerospace or automotive industries -where part failure is to be avoided at all costs.

This work offers a comprehensive overview of AM technologies and FFF in Chapter 2. Chapter 3 details the failure criterion used, as well as outlining its advantages over similar criteria. Chapters 4 through ? detail the experimental setup followed, as well as outlining noteworthy results. Finally, Conclusions and Recommendations are given in Chapter ?? in the hopes of guiding future work on the topic. %REMEMBER TO MODIFY SPECIFICS OF THE TEXT HERE!.

%Nomenclature introduced in this chapter:
\nomenclature[A]{AM}{Additive Manufacturing}% 
\nomenclature[A]{RP}{Rapid Prototyping}%
\nomenclature[A]{CAD}{Computer Aided Design}%
\nomenclature[A]{FDM}{Fused Deposition Modeling\texttrademark}
\nomenclature[A]{FFF}{Fused Filament Fabrication}%

\end{document}