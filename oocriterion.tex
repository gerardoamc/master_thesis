% oocriterion.tex
\documentclass[main.tex]{subfiles}
\begin{document}
\chapter{A Novel Failure Criterion} \label{ch:oocrit}
\epigraph{\textit{The Osswald$^2$ criterion}}

As described during Section \ref{sec:FC} of Chapter \ref{ch:bg}, currently available Failure Criteria fail to completely integrate interaction effects into the modeled failure behavior of anisotropic materials. In 2017, Paul and Tim Osswald proposed a model that attempts to overcome these limitiations \cite{Osswald2017a}. This recent failure criteria has the following characteristics:
\begin{itemize}
	\item \textbf{Tensor based and purely mathematical}: as opposed to phenomenological or mechanistic models such as the Puck or Cuntze failure criteria.
	\item \textbf{Based on the Gol'denblat-Kopnov model}
	\item \textbf{Includes stress interactions that other models neglect}
\end{itemize}

Originally titled \textquotedblleft A Strength Tensor Based Failure Criterion with Stress Interactions\textquotedblright, it will be referred in this work as the Osswald-Osswald Criterion (OOC). This chapter will describe the Gol'denblat- Kopnov model upon which the OOC is based, followed by a proper description of how this novel model implements stress interactions.

\section{The Gol'denblat-Kopnov Model}\label{sec:GKC}


% Nomenclature introduced in this chapter:
\nomenclature[A]{OOC}{Osswald-Osswald Criterion}% 


% Symbols introduced in this chapter:
%\nomenclature[S]{$\sigma$}{Axial stress \nomunit{$MPa$}}
\end{document}