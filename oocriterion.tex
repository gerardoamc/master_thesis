% oocriterion.tex
\documentclass[main.tex]{subfiles}
\begin{document}
\chapter{A Novel Failure Criterion} \label{ch:oocrit}
\epigraph{\textit{The Osswald$^2$ criterion}}

As described during Section \ref{sec:FC} of Chapter \ref{ch:bg}, currently available Failure Criteria fail to completely integrate interaction effects into the modeled failure behavior of anisotropic materials. In 2017, Paul and Tim Osswald proposed a model that attempts to overcome these limitiations \cite{Osswald2017a}. This recent failure criteria has the following characteristics:
\begin{itemize}
	\item \textbf{Tensor based and purely mathematical}: as opposed to phenomenological or mechanistic models such as the Puck or Cuntze failure criteria.
	\item \textbf{Based on the Gol'denblat-Kopnov model}.
	\item \textbf{Includes stress interactions that other models neglect}.
\end{itemize}

Originally titled \textquotedblleft A Strength Tensor Based Failure Criterion with Stress Interactions\textquotedblright, it will be referred in this work as the Osswald-Osswald Criterion (OOC). This chapter will describe the Gol'denblat-Kopnov model upon which the OOC is based, followed by a proper description of how this novel model implements stress interactions.

\section{The Gol'denblat-Kopnov Model}\label{sec:GKC}
The Gol'denblat-Kopnov Criterion (GKC) describes a mathematical function that depends on the stress state of an anisotropic material. Should the computation of this expression exceed a threshold, part failure is to be expected. To that end, a scalar function that depends on stress tensors that completely characterize the state of the material was developed \cite{Goldenblat1965}. This function is shown in Equation \ref{eq:GKCgen}, where stresses are denoted $\sigma$.

\begin{equation} \label{eq:GKCgen}
f=(F_{ij}\sigma_{ij})^\alpha + (F_{ijkl}\sigma_{ij}\sigma_{kl})^\beta + (F_{ijklmn}\sigma_{ij}\sigma_{kl}\sigma_{mn})^\gamma + ...
\end{equation}

The terms $F_{ij}$, $F_{ijkl}$ and $F_{ijklmn}$ represent second, fourth and sixth order tensors respectively. These terms of the equation depend on engineering strength parameters, such as the ultimate tensile and compressive strengths of the material in a particular load direction \cite{Osswald2017a}. Due to the complexity associated with using higher order tensors, Gol'denblat and Kopnov limited their approach to using only the second and fourth order terms. Thus Equation \ref{eq:GKCgen} is reduced to:

\begin{equation} \label{eq:GKCgenTrunc}
f=(F_{ij}\sigma_{ij})^\alpha + (F_{ijkl}\sigma_{ij}\sigma_{kl})^\beta
\end{equation}

In order to attain a linear criterion scalar function, the exponents $\alpha$ and $\beta$ were assigned values of 1 and 1/2 respectively. Finally, in plain stress scenarios, the GKC becomes:

\begin{equation} \label{eq:GKCfinal}
\begin{split}
f=F_{11}\sigma_{11} + F_{22}\sigma_{22} + F_{12}\tau_{12} + (F_{1111}\sigma_{11}^{2} + F_{2222}\sigma_{22}^{2} + F_{1212}\tau_{12}^{2} \\ + 2F_{1122}\sigma_{11}\sigma_{22} + 2F_{1112}\sigma_{11}\tau_{12} + 2F_{2212}\sigma_{22}\tau_{12})^{1/2}
\end{split}
\end{equation}

Note that in Equation \ref{eq:GKCfinal} $\sigma$ and $\tau$ denote normal and shear stresses respectively. Figure \ref{fig:loaddir} depicts an anisotropic material and all the possible loading directions for reference.

\begin{figure}[h]
	\center
	\includegraphics[height=5cm]{reference_cube}
	\caption{Different load directions in an anisotropic material} \label{fig:loaddir}
\end{figure}

Per Gol'denblat and Kopnov's design, should Equation \ref{eq:GKCfinal} be greater or equal to 1, part failure is to be expected. However, to simplify calculations, they deliberately assumed the interaction terms $F_{1112}$ and $F_{2212}$ to be zero. This is an important consideration that will come into play when describing the OOC.

Most of the terms in the GKC are obtained through mechanical testing of coupons under pure uniaxial loads in the 1 or 2 direction, or pure shear in the 1-2 plane \cite{Osswald2017a}. In these scenarios, $f$ will be equal to 1 at failure, and the stress state will be known to the user, allowing some of the unknown tensorial parameters to be easily calculated. Using $F_{11}$ and $F_{1111}$ as examples, the process would be as follows:
\pagebreak
\begin{enumerate}
	\item The tensile and compressive strength in the 1-1 direction would be obtained through mechanical testing. These values are named $X_t$ and $X_c$ respectively.
	\item Under these failure conditions, Equation \ref{eq:GKCfinal} is reduced to the following system of equations:
	\[
	\systeme*{1=F_{11}X_t + (F_{1111}X_t^{2})^{1/2}, 1= -F_{11}X_c + (F_{1111}X_c^{2})^{1/2}}
	\]

	\item $F_{11}$ and $F_{1111}$ can be obtained, yielding $F_{11}=\frac{1}{2}(\frac{1}{X_t}-\frac{1}{X_c})$ and $F_{1111}=\frac{1}{4}(\frac{1}{X_t}+\frac{1}{X_c})^2$.
\end{enumerate}

The only exception to this procedure would be the $F_{1122}$ component, which requires measuring the positive and negative shear strengths of a coupon with reinforcement oriented in 45$^\circ$. These parameters are named $S_{45p}$ and $S_{45n}$ respectively. Table \ref{tab:GKparam} summarizes the nomenclature used for the strength parameters required to completely populate the failure function of the GKC. Table \ref{tab:GKtens} summarizes all the tensorial component calculations.

\begin{table}[h]
	\centering
	\caption{Nomenclature of the GKC parameters}
	\label{tab:GKparam}
	\begin{tabu} to 0.75\textwidth {  X[c]  X[c] }
		\hline
		\textbf{Parameter} & \textbf{Description} \\ 
		\hline
		$X_t$ & Tensile strength in the 1-1 direction\\
		$X_c$ & Compressive strength in the 1-1 direction\\
		$Y_t$ & Tensile strength in the 2-2 direction\\
		$Y_c$ & Compressive strength in the 2-2 direction\\
		$S_{45p}$ & Positive shear strength for 45$^\circ$ specimen\\
		$S_{45n}$ & Negative shear strength for 45$^\circ$ specimen\\
		$S$ & Shear strength in the 1-2 plane\\
		\hline
	\end{tabu}
\end{table}

\begin{table}[h]
	\centering
	\caption{Tensorial components of the GKC}
	\label{tab:GKtens}
	\begin{tabu} to 0.8\textwidth {  X[c]  X[c] }
		\hline
		\textbf{Tensorial component} & \textbf{Formula} \\ 
		\hline
		$F_{11}$ & $\frac{1}{2}(\frac{1}{X_t}-\frac{1}{X_c})$\\
		$F_{1111}$ & $\frac{1}{4}(\frac{1}{X_t}+\frac{1}{X_c})^2$\\
		$F_{22}$ & $\frac{1}{2}(\frac{1}{Y_t}-\frac{1}{Y_c})$\\
		$F_{2222}$ & $\frac{1}{4}(\frac{1}{Y_t}+\frac{1}{Y_c})^2$\\
		$F_{12}$ & 0\\
		$F_{1212}$ & $\frac{1}{S^2}$\\
		$F_{1122}$ & $\frac{1}{8}[(\frac{1}{X_t}+\frac{1}{X_c})^2+(\frac{1}{Y_t}+\frac{1}{Y_c})^2-\frac{1}{S_{45p}}+\frac{1}{S_{45n}})^2]$\\
		\hline
	\end{tabu}
\end{table}
% Nomenclature introduced in this chapter:
\nomenclature[A]{OOC}{Osswald-Osswald Criterion}% 
\nomenclature[A]{GKC}{Gol'denblat-Kopnov Criterion}% 

% Symbols introduced in this chapter:
\nomenclature[S]{$X_t$}{Tensile strength in the 1-1 direction \nomunit{$MPa$}}
\nomenclature[S]{$X_c$}{Compressive strength in the 1-1 direction \nomunit{$MPa$}}
\nomenclature[S]{$Y_t$}{Tensile strength in the 2-2 direction \nomunit{$MPa$}}
\nomenclature[S]{$Y_c$}{Compressive strength in the 2-2 direction \nomunit{$MPa$}}
\nomenclature[S]{$S$}{Shear strength in the 1-2 plane \nomunit{$MPa$}}
\nomenclature[S]{$S_{45p}$}{Positive shear strength for 45$^\circ$ specimen \nomunit{$MPa$}}
\nomenclature[S]{$S_{45n}$}{Negative shear strength for 45$^\circ$ specimen \nomunit{$MPa$}}
\end{document}