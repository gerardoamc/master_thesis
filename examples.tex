% Examples

\documentclass[main.tex]{subfiles}
\begin{document}

\chapter{Examples}
My motivation is \cite{einstein}.
\section{A section}
More text \cite{knuthwebsite}. Numbers can be in a box \cite{Singamneni2012} as seen in table~\ref{tab:box2} on test test page~\pageref{tab:box2} or in \ref{fig:xloc}.
\begin{figure}[h]
	\centering
	\includegraphics[scale=0.4]{Xloc.jpg}
	\caption{Robot X location}
	\label{fig:xloc}
\end{figure}

\begin{table}
  \centering
    \begin{tabular}{| l c r |}
    \hline
    1 & 2 & 3 \\
    4 & 5 & 6 \\
    7 & 8 & 9 \\
    \hline
    \end{tabular}
  \caption{A simple table}
  \label{tab:box1}
\end{table}

\begin{table}
  \centering
    \begin{tabular}{| l c r |}
    \hline
    1 & 2 & 3 \\
    4 & 5 & 6 \\
    7 & 8 & 9 \\
    \hline
    \end{tabular}
  \caption{The same simple table}
  \label{tab:box2}
\end{table}

\lipsum[1-6]

\chapter{Equations}
This is how SciSlice calculates extrusion rate.
$$
ER = \frac{4 \times LH \times ND \times EF}{\pi \times FD^2}
$$
where:

$ER$ = Extrusion Rate - $mm/min$
\nomenclature[S]{$ER$}{Extrusion Rate - Speed filament in fed into the extruder \nomunit{$mm/min$}}

$LH$ = Layer Height - $mm$
\nomenclature[S]{$LH$}{Layer Height  \nomunit{$mm$}}

$ND$ = Nozzle Diameter - $mm$
\nomenclature[S]{$ND$}{Nozzle Diameter \nomunit{$mm$}}

$EF$ = Extrusion Factor
\nomenclature[S]{$EF$}{Extrusion Factor - Multiplier to adjust for over/under filling \nomunit{$-\enspace$}}

$FD$ = Filament Diameter - $mm$
\nomenclature[S]{$FD$}{Filament Diameter \nomunit{$mm$}}
 
For this project we used Fused Filament Fabrication, FFF,
\nomenclature[A]{FFF}{Fused Filament Fabrication}


\end{document}







