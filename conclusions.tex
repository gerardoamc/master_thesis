% conclusions.tex
\documentclass[main.tex]{subfiles}
\begin{document}
\chapter{Conclusions and Outlook} \label{ch:concl}
The acceptance of FFF as a formal manufacturing technique still has to overcome a plethora of issues and standardizations, including the ability to predict part failure. This works proves that applying the Osswald-Osswald failure criterion to FFF is not only possible, but can potentially serve as a tool towards achieving part failure prediction. This research also reveals a series of interesting points regarding the mechanical properties of FFF parts. First and foremost, results indicate that there is a strong interaction between axial loads, resulting in a considerable strengthening effect in bi-axial compression scenarios. The model also captures a slight interaction between shear and axial stresses in the direction perpendicular to the beads. The interaction between shear and axial stresses in the direction of raster proved different than zero, although surprisingly smaller than expected. This knowledge proves invaluable when designing FFF parts that will serve in an application that implies being subjected to important loads. However, much work could still be done in order to further build upon the concepts and results presented in this work. Below are some suggestions regarding future venues, presented in no particular order of importance.

\begin{itemize}
	\item \textbf{Develop a search algorithm for optimal interaction slopes}: At the moment, the optimal interaction slope is found through trial and error, and visual inspection of the data by the user of the OOC. The criterion could benefit from an algorithm that applies optimization techniques to find the best slope automatically and through mathematically sound and repeatable methods. 
	\item \textbf{Inquire upon the impact of process and testing parameters}: A conscious effort was made to maintain printing parameters and testing conditions as constant as possible throughout this work. However, research indicates that some mechanical properties, such as the tensile strength, are sensitive to processing parameters and testing speeds, while others like the compressive strength remain mostly unchanged. In terms of the failure surface, this would imply that some parameters could potentially be constants, which could greatly reduce the amount of work necessary to construct a failure envelope. Additionally, literature regarding the mechanical properties of FFF parts under shear stresses is nearly non-existent at the moment, so further developments in this area could prove highly beneficial to the field. 
	\item \textbf{Expand upon the probabilistic approach}: Sample production proved to be the most important bottleneck in this work. Torsion samples took on average two and a half hours to produce, requiring constant supervision of the robotic printer. This meant that in certain scenarios, sample sizes were relatively small for more serious statistical analysis. While the surface developed in this approach was calculated using average strength values, some sample populations were small enough to deter the application of safety factors in the form of confidence intervals based upon multiples of standard deviations. This could be solved by producing larger sample sizes once the difficulties associated with part production are resolved. 
\end{itemize}


\end{document}